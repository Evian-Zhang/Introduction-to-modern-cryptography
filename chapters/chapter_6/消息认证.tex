% !TEX root = ../../现代密码学简介.tex
\chapter{消息认证}
\section{简介}
我们在进行安全通信的过程中,有许多需要关注的问题。本章介绍的技巧主要与两个问题有关:
\begin{itemize}
	\item 接收者如何确认发送者的身份
	\item 接收者如何确认消息没有被篡改
\end{itemize}

这两个问题在密码学中被称作消息认证(Message Authentication)问题。\par
在上一章讨论的安全协议中,我们可以利用一些复杂的安全协议,达到甚至在没有第三方的情况下,仍然能确认发送者的身份。但是,我们到目前为止却并没有找到能保证消息没有被篡改的方法。\par
因此,我们首先解决第二个问题:如何确认消息没有被篡改。
\section{哈希函数}
\subsection{定义}
在讨论哈希函数之前,我们需要充分讨论一下目前需要完成的目标:接收者如何确认消息没有被篡改。如果一个算法要实现这个目标,那么这个算法必然是接收我们要传递的消息$m$,并产生一个输出$h$。发送者利用这个算法,将消息$m$以及输出$h$都发送给接收者,而接收者再次利用这个算法处理收到的消息$m'$,通过对比输出$h'$与发送者发送的输出$h$,判断消息$m$是否得到篡改。那么,这一算法最重要的一点是什么呢?显然是尽量避免两个不同的消息$m_1\neq m_2$, 其通过这个算法产生的输出$h_1$和$h_2$相同。如果产生这种情况,我们则称该算法拥有一个\textbf{碰撞}。更进一步地,从抵抗攻击的角度来看,这种算法应该使敌手在截获消息经过该算法的输出$h$以后,不能或者难以生成一个篡改的消息$m'$, 使得其经过该算法的输出$h'=h$. 在最理想的状态下,实现这一目标最好的方法就是,从$h$中敌手无法获得任何有关$m$的信息。\par
根据以上的讨论,这个算法$H(x)$需要满足的性质有:
\begin{itemize}
	\item 可以适用于任意长度的消息$x$
	\item 算法的输出$H(x)$是固定长的
	\item 由输入$x$求得输出$H(x)$的过程较容易,可用硬件和软件实现
	\item 由输出$H(x)$反求得输入$x$的过程是不可行的,或者是难以计算的
	\item 给定$x$, 难以求得$y$, 使得$H(x)=H(y)$. 这一点被称作\textbf{弱碰撞抵抗}
	\item 难以求得任何一对$\pth{x, y}$, 使得$H(x)=H(y)$. 这一点被称作\textbf{强碰撞抵抗}
\end{itemize}

对于第二点,即算法的输出是固定长的。这是由于上述讨论中我们提到,从输出中敌手无法获得输入的任何信息,这里甚至包括输入的消息的长度。不仅如此,在实际操作中,由于“由输入求得输出的过程较容易”这一要求,如果输出是固定长的,那么程序中内存分配等问题就相对较容易。因此,现在通行的算法的输出都是固定长度的。同时,我们也应该了解到,这个“固定长”的长度一般是小于消息的长度的。\par
用于确认消息没有篡改的算法被称为哈希函数(Hash fucntion). 又称为散列函数。其接受一个消息作为输入,产生的输出被称为该消息的摘要(digest).\par
除了上述功能之外,由于其输出的长度是固定的,因此,哈希函数还有一个重要的用途:当我们要处理的数据很大时,将其通过哈希函数,压缩成较小的数据,则更好处理。
\subsection{Merkle-Damg\aa rd结构}
在密码学中,许多哈希函数都基于一个类似的结构——Merkle-Damg\aa rd结构。首先,我们先介绍一下该结构的两个主要部分。
\subsubsection{服从MD结构的填充函数}
对于正整数$b$来说,如果将所有长度为$b$的整数倍的二进制串构成的集合记作$B$, 将所有长度小于$2^b$的二进制串构成的集合记作$U$. 同时,对于长度为$b$的整数倍的二进制串$M$,将其从头开始分割为长度均为$b$的二进制子串,那么称每个子串为$M$的一个块。\par
那么对于非空集合$D\subset U$, 以及函数$\mathrm{Pad}:D\to B$, 称满足以下条件的函数$\mathrm{Pad}$为一个相对于$b$的服从MD结构的填充函数:
\begin{itemize}
	\item 对于任意$M\in D$, $M$是二进制串$\mathrm{Pad}\pth{M}$的前缀
	\item 对于任意$M_1, M_2\in D$, 如果$M_1$和$M_2$的长度相同,那么$\mathrm{Pad}\pth{M_1}$和$\mathrm{Pad}\pth{M_2}$的长度相同
	\item 对于任意$M_1, M_2\in D$, 如果$M_1$和$M_2$的长度不同, 那么$\mathrm{Pad}\pth{M_1}$和$\mathrm{Pad}\pth{M_2}$的最后一个块不同
\end{itemize}

对于任意正整数$b$, 我们可以轻易地构造出服从MD结构的填充函数:对于长度为$k$的二进制串$M$, 如果$k\bmod b\neq b-1$, 那么将二进制串$k$后不停地填充$0$, 直到新串的长度模$b$为$b-1$. 然后,将最初串$M$的长度$k$的二进制表示的最后一位填充在最后,形成一个长度为$b$的整数倍的串。可以轻松地证明,这种填充方法,是服从MD结构的。\par
在上述构造方法中,我们提到,在新串的末尾填充最初串的长度,这一方法被称为Merkle-Damg\aa rd强化。
\subsubsection{单向压缩函数}
所谓单向压缩函数,其类似于输入是固定长度的哈希函数。对于单向压缩函数$h$, 其接受两个参数,一个为长度为$b$的二进制串,一个为长度为$v$的二进制串,并输出一个长度为$b$的二进制串。这可以看成一个输入长度为$b+v$, 输出长度为$b$的哈希函数。类似于哈希函数,单向压缩函数也有碰撞抵抗等要求。
\subsubsection{算法}
要实现服从Merkle-Damg\aa rd结构的输出为$b$比特的哈希函数$H$,需要两个正整数$b, v$. 一个相对于$v$的使用了Merkle-Damg\aa rd强化的服从MD结构的填充函数$\mathrm{Pad}$, 一个接受参数为分别为长度为$b$和$v$的二进制串的单向压缩函数$h$, 以及一个长度为$b$的初始值$\mathrm{IV}$.\par
那么,Merkle-Damg\aa rd结构的算法为:
\begin{enumerate}
	\item 将输入的二进制串$M$用$\mathrm{Pad}$函数填充为长度为$nb$比特的二进制串$M'=\mathrm{Pad}\pth{M}$
	\item 将$M'$从头开始分割为$n$个长度为$b$比特的子串$M_1, M_2, \ldots, M_n$
	\item 对于每个子串$M_i$, 计算$H_i=h\pth{M_i, H_{i-1}}$, 其中$H_0=\mathrm{IV}$
	\item $H$的输出即为$M_n$
\end{enumerate}

同时,Merkle和Damg\aa rd分别独立证明出来,如果单向压缩函数$h$能抵抗碰撞,那么$H$也能抵抗碰撞。\par
接下来,介绍两个基于Merkle-Damg\aa rd结构的哈希算法:MD5和SHA-1
\subsection{MD5}
MD5哈希算法的输出为128比特的二进制串,其$b=128, v=512$. 然而,MD5算法并不安全。在2012年,MD5算法被卡耐基梅隆大学软件工程研究所宣布能被密码学方法破解,不适合将来的使用。但是,如果只是简单的摘要对比,对安全性并不高的话,MD5哈希算法也是一个比较简便的方法。\par
在MD5算法中,二进制串既要以比特串的形式进行异或等逻辑运算,也要以二进制数的形式进行加减等算术运算。因此,在MD5算法中涉及到了数据的存储问题。由于是在计算机上实现的,而计算机中一个数据单元是1字节,也就是8比特。因此,数据存储的问题变成了字节存储顺序的问题。比如说,以十六进制表示的二进制串$1234$究竟表示的是$1\times 16^3+2\times 16^2+3\times 16+4$还是$4\times 16^3+3\times 16^2+2\times 16+1$呢?前者被称为小端法,后者被称为大端法(但无论是小端法还是大端法,每个字节内部比特的存储顺序都是小端法)。而MD5算法中使用的是大端法。但是,现代计算机以及本算法使用的\verb`BigInteger`类均采用小端法处理二进制串。因此,在C++程序实现的时候,每次讲二进制串转化为数字,以及数字转化为二进制串时,都要调整端序。\par
虽然事实上,端序并不会影响加密性,但是,作为一个通行的标准,MD5码仍然需要我们使用大端法。\par
MD5算法包括了四个步骤:
\begin{enumerate}
	\item 填充比特
	\item 填充长度
	\item 初始化链接变量
	\item 处理消息
\end{enumerate}

其中,填充比特和填充长度就是之前讲的Merkle-Damg\aa rd结构中的填充函数,而初始化链接变量是为了之后处理消息使用的参数,处理消息则是运用单向压缩函数,并进行输出。
\subsubsection{填充比特}
假设输入消息的长度为$k$, 那么如果$k\bmod{512}<448$, 那么将其后先填充一个'1', 再不停填充'0'直到新串的长度模$512$等于$448$. 如果$k\bmod{512}>448$, 那么先将其后填充一个'1', 然后再不停填充0,填满一个块以后,再继续填充$448$位。\par
其C程序实现如下:
\begin{prove}
\begin{verbatim}
BigInteger appendPaddingBits(BigInteger plainText)
{
    int spareLength = plainText.getLength() % 512;
    if (spareLength < 448)
    {
        int appendLength = 448 - spareLength;
        bool *appendValue = new bool[appendLength];
        *(appendValue + appendLength - 1) = true;
        int index = appendLength - 2;
        while (index > -1)
        {
            *(appendValue + index) = false;
            index--;
        }
        plainText.append(appendValue, appendLength);
    }
    else if (spareLength > 448)
    {
        int appendLength = 512 - spareLength + 448;
        bool *appendValue = new bool[appendLength];
        *(appendValue + appendLength - 1) = true;
        int index = appendLength - 2;
        while (index > -1)
        {
            *(appendValue + index) = false;
            index--;
        }
        plainText.append(appendValue, appendLength);
    }
    return plainText;
}
\end{verbatim}
\end{prove}
\subsubsection{填充长度}
记最初消息的长度为$k$. 将其表示成二进制串,如果其长度小于64位,则再其首部填0;如果其长度大于64位,则截取其最后64位。\par
这里由于涉及到数向二进制串的转化,需要调整端序。\par
其C程序实现如下:
\begin{prove}
\begin{verbatim}
BigInteger appendLength(BigInteger appendedText, int textLength)
{
    BigInteger appendBigInt(textLength);
    appendBigInt.limitTo(64);
    BigInteger bitSlice[8];
    for (int byte = 0; byte < 8; byte++)
    {
        bitSlice[byte] = appendBigInt.slice(8 * byte, 
                                            8 * (byte + 1));
        appendedText.append(bitSlice[byte]);
    }
    return appendedText;
}
\end{verbatim}
\end{prove}
\subsubsection{初始化链接变量}
在MD5最初使用的大端法表示下,MD5使用的4个32位链接变量的值分别是:
\begin{itemize}
	\item \verb`A=0x01234567`
	\item \verb`B=0x89abcdef`
	\item \verb`C=0xfedcba98`
	\item \verb`D=0x76543210`
\end{itemize}

但是,由于这也涉及到数向二进制串的转化,因此,需要调整端序。\par
其C程序实现如下:
\begin{prove}
\begin{verbatim}
BigInteger A, B, C, D;
void initializeMDBuffer()
{
    A = BigInteger("0x67452301");
    B = BigInteger("0xefcdab89");
    C = BigInteger("0x98badcfe");
    D = BigInteger("0x10325476");
}
\end{verbatim}
\end{prove}
\subsubsection{处理消息}
在处理消息的过程中,需要使用四个辅助函数\verb`FF`, \verb`GG`, \verb`HH`, \verb`II`. 而这四个辅助函数又是由另外四个辅助函数\verb`F`, \verb`G`, \verb`H`, \verb`I`定义的。因此,我们首先定义八个函数:
\begin{prove}
\begin{verbatim}
BigInteger F(BigInteger X, BigInteger Y, BigInteger Z)
{
    return (X & Y) | (~X & Z);
}

BigInteger G(BigInteger X, BigInteger Y, BigInteger Z)
{
    return (X & Z) | (Y & ~Z);
}

BigInteger H(BigInteger X, BigInteger Y, BigInteger Z)
{
    return (X ^ Y) ^ Z;
}

BigInteger I(BigInteger X, BigInteger Y, BigInteger Z)
{
    return Y ^ (X | ~Z);
}

BigInteger X[16];
BigInteger T[64];

void FF(BigInteger &A, BigInteger B, BigInteger C, BigInteger D, 
        int k, int s, int i){
    BigInteger tmp = A + F(B, C, D) + X[k] + T[i];
    tmp.limitTo(32);
    A = B + tmp.rotLeft(s);
    A.limitTo(32);
}

void GG(BigInteger &A, BigInteger B, BigInteger C, BigInteger D, 
        int k, int s, int i)
{
    BigInteger tmp = A + G(B, C, D) + X[k] + T[i];
    tmp.limitTo(32);
    A = B + tmp.rotLeft(s);
    A.limitTo(32);
}

void HH(BigInteger &A, BigInteger B, BigInteger C, BigInteger D, 
        int k, int s, int i)
{
    BigInteger tmp = A + H(B, C, D) + X[k] + T[i];
    tmp.limitTo(32);
    A = B + tmp.rotLeft(s);
    A.limitTo(32);
}

void II(BigInteger &A, BigInteger B, BigInteger C, BigInteger D, 
        int k, int s, int i)
{
    BigInteger tmp = A + I(B, C, D) + X[k] + T[i];
    tmp.limitTo(32);
    A = B + tmp.rotLeft(s);
    A.limitTo(32);
}
\end{verbatim}
\end{prove}