% !TEX root = ../../现代密码学简介.tex
\chapter{消息验证码}
\section{简介}
通过上一章的介绍,我们掌握了一个用于验证消息完整性的强大工具:哈希函数。通过比较发送者发送的摘要与接收者计算的摘要,接收者可以判断消息是否在传递过程中受到篡改。但是,这仅仅是第一步。在消息验证的过程中,还有一个非常重要的目的:在确定消息完整性的同时,确定消息发送自期望的发送者。因此,我们引入一个新的工具:消息验证码(Message Authentication Code, MAC).\par
在之前提到安全协议的时候,我们知道,密钥交换协议可以确保消息发送自期望的发送者。但是,密钥交换协议却不能确定消息的完整性。那么,如何同时做到确定消息的完整性,和确定消息发送自期望的发送者呢?\par
在之后提到的安全协议中,我们可以了解到,通过某些协议可以确保消息发自期望的发送者,那么换个角度说,如果接收者能确定消息的发送者和自己使用的是相同的密钥,那么也就能确定消息的发送者是期望的发送者。因此,我们可以结合密钥和之前提到的哈希函数的基本概念,从而提出MAC算法$C_K(m)$的基本概念:
\begin{itemize}
	\item 该算法适用于任意长度的消息$m$
	\item 算法使用一个密钥$K$
	\item 该算法的输出$C_K(m)$是固定长的
	\item 由输入$m$和密钥$K$求得输出$C_K(m)$的过程较容易
	\item 由输出$C_K(m)$反求得输入$m$或密钥$K$的过程是不可行的,或者是难以计算的
	\item 给定$x$, 难以求得$y$, 使得$C_K(x)=C_K(y)$
	\item 难以求得任何一对$\pth{x, y}$, 使得$C_K(x)=C_K(y)$
\end{itemize}

我们可以看到,MAC算法的设计准则和哈希算法的设计准则几乎一致,只是MAC算法还需要一个密钥。此外,MAC算法的输出称为\textbf{标签}(tag).\par
发送方通过密钥$K$, 使用MAC算法对要传递的消息$m$进行处理,得到标签$t$, 将$\pth{m, t}$发送给接收者。接收方接收到$\pth{m', t}$后,通过与发送方共享的密钥$K$, 使用同样的MAC算法对接收到的消息$m'$进行处理,得到标签$t'$. 如果$t'=t$, 那么既说明了发送方发送的消息并没有得到篡改,也说明了发送方与自己使用的是同一个密钥,也就是消息的发送方是期待的发送方。\par
上述过程虽然并不是真的符合认证加密的过程,但是讲述了MAC的基本思想。\par
MAC最主要的两个部分,一个是密钥,一个是类似哈希函数的基本概念。因此,MAC主要的实现方法,也分为两类:基于分组密码,与基于哈希函数。
\section{基于分组密码的MAC算法}
\subsection{CBC-MAC}
我们之前在分组密码中提到,分组密码有许多运行模式。CBC-MAC就是基于密码分组连接模式(CBC)的MAC算法。其也被称为数据认证算法(Data Authentication Algorithm). 但是由于其安全性欠佳,因此已被淘汰。\par
假设要传递的消息为$M$, 密钥为$K$, 分组密码加密算法为$\E{k}{m}$, 其分组长度为$b$比特。\par
CBC-MAC的过程如下:
\begin{enumerate}
	\item 填充\par
	将消息$M$通过在末尾补`0'填充为长度是$b$的整数倍$nb$的二进制串$M'$, 并将其分为$n$个长度为$b$比特的二进制串$m_1, m_2, \ldots, m_n$.
	\item 加密\par
	采用以下公式进行加密:
	\begin{align*}
	&O_1=\E{K}{m_1}\\
	&O_{i}=\E{K}{O_{i-1}\oplus m_{i}},\quad i=2, 3,\ldots, n
	\end{align*}
	\item 输出\par
	CBC-MAC的输出为$b$比特二进制串$O_n$
\end{enumerate}

从本质上来说,这个算法就是一个初始值$\mathrm{IV}=0$的CBC模式分组加密算法,取其最后一个分组的输出。\par
对于固定长度的消息来说,如果使用的分组加密算法是安全的,那么CBC-MAC也是安全的。但是,对于长度不固定的消息来说,这是不安全的。
\subsection{CMAC}
CMAC是NSIT推荐的基于分组加密算法的MAC算法。\par
假设要传递的消息为$M$, 密钥为$K$, 分组密码加密算法为$\E{k}{m}$, 其分组长度为$b$比特。\par
CMAC分为两个步骤:生成子密钥与输出标签。
\subsubsection{生成子密钥}
对于密钥$K$, 采用下面的算法将其生成为两个长度为$b$比特的子密钥$K_1, K_2$
\begin{enumerate}
	\item 计算$b$位二进制数$K_0=\E{K}{0}$
	\item 记$B_0$为$K_0$表示成二进制数时的最高位(第$b$位)。\par
	如果$B_0=0$, 则$K_1=K_0\ll 1$, 其中$x\ll 1$表示将二进制串$x$左移一个比特;\par
	如果$B_0=1$, 则$K_1=\pth{K_0\ll 1}\oplus C$, 其中
	\[C=\begin{dcases}\pth{1B}_{16}&b=64\\\pth{87}_{16}&b=128\\\pth{425}_{16}&b=256\end{dcases}\]
	\item 记$B_1$为$K_1$表示成二进制数时的最高位。\par
	如果$B_1=0$, 则$K_2=K_1\ll 1$.\par
	如果$B_1=1$, 则$K_2=\pth{K_1\ll 1}\oplus C$
\end{enumerate}
\subsubsection{输出标签}
CMAC采用如下算法产生$b$比特的标签:
\begin{enumerate}
	\item 将$M$从头开始分为$n$个子串$m_1, m_2, \ldots, m_n$. 其中$m_1, m_2,\ldots, m_{n-1}$的长度均为$b$比特,$m_n$的长度不大于$b$比特。
	\item 如果$m_n$的长度是$b$比特,那么$m_n'=K_1\oplus m_n$.\par
	如果$m_n$的长度小于$b$比特,那么先在其后补一位`1', 然后一直补`0'直到其长度达到$b$比特的二进制串$m_n''$, 然后$m_n'=K_2\oplus m_n''$.
	\item 采用如下公式加密:
	\begin{align*}
	&O_1=\E{K}{m_1}\\
	&O_{i}=\E{K}{O_{i-1}\oplus m_{i}},\quad i=2, 3,\ldots, n\\
	&O_n=\E{K}{O_{n-1}\oplus m_n'}
	\end{align*}
	\item 输出\par
	CMAC的输出为$b$比特二进制串$O_n$
\end{enumerate}
\section{基于哈希函数的MAC算法}
除了基于分组密码的MAC算法,还有一种是基于哈希函数的MAC算法,称为HMAC.\par
HMAC需要一个哈希函数$H(x)$, 根据我们上一章的记号,假设其输出为$b$比特,输入被处理为$v$比特的块。同时,HMAC还需要一个密钥$K$. 对$K$的长度有如下要求:
\begin{itemize}
\item $K$的长度不建议小于$b$比特。
\item $K$的长度如果大于$v$比特,那么就对$K$进行哈希得到$b$比特长的串$H\pth{K}$作为实际使用的密钥$K$
\end{itemize}

通过上述操作,密钥$K$的长度被控制在不大于$v$比特。接着,在二进制串$K$的尾部填充`0'使得其长度达到$v$比特的新串$K'$.\par
接下来需要定义两个串$\mathrm{ipad}$和$\mathrm{opad}$:
\begin{itemize}
	\item $\mathrm{ipad}$是将十六进制串$36$重复多次得到的长度为$v$的二进制串
	\item $\mathrm{opad}$是将十六进制串$5C$重复多次得到的长度为$v$的二进制串
\end{itemize}

那么,HMAC算法的输出$C_K(m)$的定义为
\begin{equation}
	C_K(m)=H\pth{K'\oplus\mathrm{opad}\parallel H\pth{K'\oplus\mathrm{ipad}\parallel m}}
\end{equation}
其中$x\parallel y$代表将二进制串$y$级联在$x$尾部。
\section{基于全域哈希的MAC算法}
基于全域哈希(Universal hashing)的MAC算法,简称为UMAC. 它的特点是,对于每一次求MAC的消息,其采用的哈希函数是不同的。也就是对于同一个消息,多次求MAC的结果并不相同。现在最常用的UMAC算法,就是Poly1305, 其也被称为目前处于顶尖水平的MAC算法。这种算法运算速度极快,安全性也较高。下面就介绍一下该算法。
\subsection{Poly1305}
该算法接收任意长度的消息$M$,一个256比特的密钥$K$, 输出为128比特的标签。该算法要求使用的密钥每一次都不同,因此,属于全域哈希。\par
该算法涉及到二进制串与二进制数的转化,因此,需要涉及端序的问题。而该算法属于小端法,因此需要调整端序。\par
该算法分为如下几个步骤:
\begin{enumerate}
	\item 处理密钥
	\item 设置常数
	\item 更新累加器
	\item 输出
\end{enumerate}
\subsubsection{处理密钥}
对于输入的256位密钥$K$, 将其分为左右两个128比特的串$r$和$s$(这里要求$r$和$s$都需要调整端序). 然后对于$r$, 需要对其进行过滤(clamp). 要求其调整端序后,从右往左数:
\begin{itemize}
	\item 第$4, 8, 12, 16$个字节的高4位变为0
	\item 第$5, 9, 13$个字节的最低2位变为0
\end{itemize}

其C程序实现如下:
\begin{prove}
	\begin{verbatim}
void clampR(BigInteger &r)
{
    r &= BigInteger::bigIntegerFromHexString("0ffffffc0ffffffc
                                              0ffffffc0fffffff");
}

BigInteger r = key.slice(0, 128);
changeEndian(r);
clampR(r);
BigInteger s = key.slice(128, 256);
changeEndian(s);
	\end{verbatim}
\end{prove}
\subsubsection{设置常数}
在该算法中,涉及到常数$p=2^{130}-5$. 同时,将累加器\verb`accumulator`置为0.\par
其C程序实现如下:
\begin{prove}
	\begin{verbatim}
BigInteger accumulator = 0;
BigInteger p = (BigInteger(1) << 130) - 5;
	\end{verbatim}
\end{prove}
\subsubsection{更新累加器}
将消息$M$从左向右分为16字节,也就是128比特的块(最后一部分可以小于128比特)。然后对于每个块,进行如下操作以更新累加器:
\begin{enumerate}
	\item 将该块以小端法转化为二进制数$m$
	\item 在二进制数$m$的最高位前面加`00000001', 也就是说,如果$m$的十六进制表示为\\$\pth{22222222222222222222222222222222}_{16}$, 那么加了之后的十六进制表示为\\$\pth{0122222222222222222222222222222222}_{16}$
	\item 如果该块的长度仍然小于17字节(即对于最后一个块来说),那么只需要在其高位填充0,使其达到17字节即可
	\item 将最终得到的二进制数$m$加到\verb`accumulator`上
	\item 将\verb`accumulator`乘$r$后再模$p$
\end{enumerate}

其C程序实现如下:
\begin{prove}
	\begin{verbatim}
void processBlock(BigInteger &block)
{
    const int N = block.getLength() / 8;
    changeEndian(block);
    BigInteger addByte = BigInteger(1) << block.getLength();
    block += addByte;
    if (block.getLength() < 136)
    {
        block.limitTo(136);
    }
}

int length = plainText.getLength();
int N = length / 128;
if (length % 128 != 0)
	N++;
for (int i = 0; i < N; i++)
{
	int upper = (i + 1) * 128;
	if (upper > length)
		upper = length;
	BigInteger block = plainText.slice(i * 128, upper);
	processBlock(block);
	accumulator += block;
	accumulator = (r * accumulator) % p;
}
accumulator += s;
	\end{verbatim}
\end{prove}
\subsubsection{输出}
将\verb`accumulator`转化成二进制串,也就是小端法调整端序后,输出其最低的128位即可。\par
其C程序实现如下:
\begin{prove}
	\begin{verbatim}
BigInteger tag = accumulator;
tag.limitTo(128);
changeEndian(tag);
	\end{verbatim}
\end{prove}
\subsubsection{总过程}
综上,Poly1305的总过程的C程序实现如下:
\begin{prove}
	\begin{verbatim}
BigInteger Poly1305_MAC(BigInteger plainText, BigInteger key)
{
    BigInteger r = key.slice(0, 128);
    changeEndian(r);
    clampR(r);
    BigInteger s = key.slice(128, 256);
    changeEndian(s);
    BigInteger accumulator = 0;
    BigInteger p = (BigInteger(1) << 130) - 5;
    int length = plainText.getLength();
    int N = length / 128;
    if (length % 128 != 0)
        N++;
    for (int i = 0; i < N; i++)
    {
        int upper = (i + 1) * 128;
        if (upper > length)
            upper = length;
        BigInteger block = plainText.slice(i * 128, upper);
        processBlock(block);
        accumulator += block;
        accumulator = (r * accumulator) % p;
    }
    accumulator += s;
    BigInteger tag = accumulator;
    tag.limitTo(128);
    changeEndian(tag);
    return tag;
}
	\end{verbatim}
\end{prove}
\section{认证加密}
通过MAC, 我们可以实现确定消息的完整性,和确定消息源头的真实性。但是,MAC做不到的却是提供保密性。因此,在实际的安全通信过程中,需要我们同时确保保密性、完整性与真实性。因此,在实际应用中,往往使用的是加密算法与MAC算法结合的方式。这种方式,称为认证加密(Authenticated encryption).\par
常见的认证加密方式有三种:先加密再MAC、加密同时MAC、先MAC再加密。\par
接下来假定由$A$向$B$发送消息$M$, 使用的加密算法为$\E{K}{m}$, MAC算法为$C_K(m)$. 加密密钥为$K_1$, MAC密钥为$K_2$. 加密密钥和MAC密钥不一定不同,这基于不同的认证加密模式。
\subsection{先加密再MAC}
先加密再MAC(Encrypt-then-MAC, EtM), 指的是先对明文加密,然后根据得到的密文生成MAC, 密文和他的MAC一起发送。即:
\[A\to B: \E{K_1}{M}, C_{K_2}\pth{\E{K_1}{m}}\]

这一方法是现在最常用的方法。
\subsection{加密同时MAC}
加密同时MAC(Encrypt-and-MAC, E\&M), 指的是基于明文生成MAC, 并且明文在没有MAC的情况下被加密。明文的MAC和密文一起发送。即:
\[A\to B:\E{K_1}{M}, C_{K_2}\pth{M}\]
\subsection{先MAC再加密}
先MAC再加密(MAC-then-Encrypt, MtE), 指的是基于明文生成MAC, 然后将明文和MAC一起加密以基于两者生成密文,密文被发送。即:
\[A\to B:\E{K_1}{M\parallel C_{K_2}\pth{M}}\]
\subsection{认证加密模式}
由于E\&M, MtE均有安全上的隐患,所以现在最常用的认证加密方法就是EtM. 但是,EtM只是规定了认证加密的方式,对具体的加密算法和MAC算法并没有规定。因此,许多具体规定了加密算法与MAC算法的认证加密模式就出现了,如OCB, CCM, EAX, GCM等。