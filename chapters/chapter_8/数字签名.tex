% !TEX root = ../../现代密码学简介.tex
\chapter{数字签名}
\section{简介}
通过上一章讲到的MAC以及认证加密,我们有效地解决了消息认证的前两个问题:完整性与真实性。接下来,我们介绍一种强大的工具——数字签名。这种工具能同时解决完整性、真实性与不可否认性这三个消息认证的问题。下面这张著名的表格归纳了消息认证的三个手段与其特点:
\begin{table}[H]
    \centering
    \begin{tabular}{c|c|c|c}\hline
        &哈希函数&MAC&数字签名\\\hline
        完整性&是&是&是\\\hline
        真实性&否&是&是\\\hline
        不可否认性&否&否&是\\\hline
        密钥类型&无密钥&对称密钥&非对称密钥\\\hline
    \end{tabular}
\end{table}

数字签名需要第三方来进行验证。那么,我们之前了解到的工具中有什么东西是第三方知道的呢?我们最先想到的就是公钥密码体系中发送者的公钥。由于涉及到了密钥,因此,数字签名总共有三个部分:
\begin{itemize}
    \item 密钥生成算法
    \item 签名算法
    \item 验证签名算法
\end{itemize}
\section{数字签名的基本步骤}
下面假设发送者是$A$, 接收者是$B$, 第三方是$C$, 要发送的消息为$M$, 采用的加密密钥为$K$, 哈希算法为$H$
\subsection{密钥生成算法}
由发送者生成一对公钥-私钥对$\pth{PK, SK}$,同时向外公布公钥$PK$。\par
注意,发送者用于签名的公-私钥对不能用于加密。
\subsection{签名算法}
\[A\to B:\E{K}{M},\]

