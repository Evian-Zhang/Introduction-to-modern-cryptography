% !TEX root = ../../现代密码学简介.tex
\chapter{流密码}
\section{数学上的基本概念}
从这章开始,我们就不再拘泥于古代的加密拉丁字母的加密方式,而开始转向数字化。我们知道,在现代科技中,密码学最常应用的领域就是计算机领域,而计算机领域则是由二进制的$0, 1$串构成。因此,在接下来的讨论中,我们都将明文空间及密文空间看作由$0$和$1$构成的二进制串组成的集合。\par
因此,为了接下来能更顺畅地进行关于流密码的讨论,这里先介绍一些数学上关于这方面的基础知识。
\subsection{$\GF\pth{2}$上的加法与乘法}
由于我们讨论的仅是$0$和$1$及它们的运算, 因此,我们定义一个有限域$\GF\pth{2}=\brace{0, 1}$. \par
有限域$\GF\pth{2}$上的加法被定义成逻辑上的异或,也可以理解成模$2$加法。以下为$\GF\pth{2}$上的加法表:
\begin{table}[H]
    \centering
    \caption{$\GF\pth{2}$上的加法表}
    \begin{tabular}{c|cc}\hline
        $+$&0&1\\\hline
        0&0&1\\
        1&1&0\\\hline
    \end{tabular}
\end{table}

如果我们定义一个集合中元素$a$的加法逆元$-a$满足
\[a+\pth{-a}=\pth{-a}+a=0\]

则$\GF\pth{2}$上的加法逆元表为:
\begin{table}[H]
    \centering
    \caption{$\GF\pth{2}$上的加法逆元表}
    \begin{tabular}{c|cc}\hline
        $a$&0&1\\\hline
        $-a$&0&1\\\hline
    \end{tabular}
\end{table}

而$\GF\pth{2}$上的减法则可以定义成与加法逆元的加法,即
\begin{equation}
    a-b=a+\pth{-b}
\end{equation}

$\GF\pth{2}$上的乘法则被定义成了逻辑上的与。以下为$\GF\pth{2}$上的乘法表:
\begin{table}[H]
    \centering
    \caption{$\GF\pth{2}$上的乘法表}
    \begin{tabular}{c|cc}\hline
        $\cdot$&0&1\\\hline
        0&0&0\\
        1&0&1\\\hline
    \end{tabular}
\end{table}

根据上述的定义,我们可以得出以下$\GF\pth{2}$上常用的运算规则:
\begin{itemize}
    \item 
        \begin{equation}
            \forall x\in\GF\pth{2}, x+x=0
        \end{equation}
    \item 
        \begin{equation}
            \forall x, y\in\GF\pth{2}, x-y=y-x=x+y
        \end{equation}
    \item 
        \begin{equation}
            \forall x\in\GF\pth{2}, x\cdot x=x
        \end{equation}
\end{itemize}
\subsection{$\GF\pth{2}$上的多项式}
此外,还有一个我们未曾接触过的知识:$\GF\pth{2}$上的多项式。为此,不妨先介绍一下多项式理论。\par
对于表达式
\begin{equation}
    a_nx^n+a_{n-1}x^{n-1}+\cdots +a_1x+a_0=\sum_{i=0}^na_ix_i
\end{equation}

我们称其系数为$a_0, a_1,\ldots, a_n$. 其系数集$S$为$a_0, a_1,\ldots ,a_n$的取值范围。当$a_n\neq 0$时,称该多项式为$S$上的$n$次多项式。比如说,如果其系数的取值仅限于$0$和$1$, 则称这个多项式为$\GF\pth{2}$上的多项式。\par
值得指出的是,我们研究多项式理论时,多项式在我们眼中仅仅是一个表达式,我们并不需要去对每一个$x$的取值进行多项式的求值。它就相当于一个集合的元素,一个多项式就是一个最小的单位。其加、减、乘、除就应该像我们定义复数集那样重新地进行定义。也就是说,我们通过定义来确定多项式$f+g=h$, 而不是通过$\forall x, h(x)=f(x)+g(x)$来定义多项式之和。尽管结果确实如此,但这应该理解为自洽的定义,而非推导。\par
\subsubsection{多项式的加法}
两个多项式之和的多项式的系数,等于其对应系数之和。即:\par
若$m\geq n$, 则
\begin{equation}
    \sum_{i=0}^ma_ix^i+\sum_{j=0}^nb_jx^j=\sum_{k=0}^n\pth{a_k+b_k}x^k+\sum_{i=n+1}^ma_ix^i
\end{equation}
其中$a_k+b_k$的加法应理解成系数集$S$上的加法。\par
用更形象的方法来说,我们不妨考虑实数集上的多项式$f=x^2+1$与$g=x^3+x^2$, 那么其和我们可以类似于小学时的竖式来计算:
\begin{table}[H]
    \centering
    \begin{tabular}{ccrcrcr}
        $f$&$=$&&&$x^2$&$+$&$1$\\
        $g$&$=$&$x^3$&$+$&$x^2$&$+$&$1$\\\hline
        $f+g$&$=$&$x^3$&$+$&$2x^2$&$+$&$2$
    \end{tabular}
\end{table}

但是,我们这里也需要注意到刚刚说的,$a_k+b_k$的加法应理解成系数集$S$上的加法。比如说还是刚刚的两个多项式,但其在$\GF\pth{2}$上的加法为:
\begin{table}[H]
    \centering
    \begin{tabular}{ccrcrcr}
        $f$&$=$&&&$x^2$&$+$&$1$\\
        $g$&$=$&$x^3$&$+$&$x^2$&$+$&$1$\\\hline
        $f+g$&$=$&$x^3$&&&&
    \end{tabular}
\end{table}

这里$2x^2$与$2$之所以不见了,是因为在$\GF\pth{2}$上,$1+1=0$.
\subsubsection{多项式的减法}
类似于多项式的加法的定义,两个多项式之差的系数,等于其对应系数之差。在$\GF\pth{2}$上的多项式之差的例子:
\begin{table}[H]
    \centering
    \begin{tabular}{ccrcrcr}
        $f$&$=$&&&$x^2$&$+$&$1$\\
        $g$&$=$&$x^3$&$+$&$x^2$&$+$&$1$\\\hline
        $f-g$&$=$&$x^3$&&&&
    \end{tabular}
\end{table}

这里是由于$\pth{0-1}x^3=\pth{0+1}x^3=x^3$, $\pth{1-1}x^2=0$, $1-1=0$.
\subsubsection{多项式的乘法}
我们可以形式上地用乘法分配律进行计算,并且约定$x^ix^j=x^{i+j}$. 下面演示一下$\GF\pth{2}$上多项式的乘法运算:
\begin{align*}
    &\pth{x+1}\pth{x+1}\\
    =&xx+\pth{1+1}x+1\\
    =&x^2+1
\end{align*}
注意到在$\GF\pth{2}$上,$\pth{1+1}x=0$.\par
同时,我们记$f^n=f\cdot f^{n-1}$.
\subsubsection{多项式的整除}
对于多项式$f$和$g$, 如果存在多项式$h$, 使得$f=gh$, 则称$g$整除$f$, 记作$g\mid f$, 同时$\ext \frac{f}{g}=h$.\par
注意这仍是在系数集$S$上的。比如说,在$\GF\pth{2}$中,我们有
\[\pth{x^2+1}\mid\pth{x+1}^2\]

此外,若一个多项式$f$是不可约的,则说明不存在两个多项式$g, h$, 使得$gh=f$且$g, h$的次数均小于$f$的次数。
\subsubsection{$\GF\pth{2}$上的常用等式}
加法交换律
\begin{equation}
    f+g=g+f
\end{equation}

加法结合律
\begin{equation}
    \pth{f+g}+h=f+\pth{g+h}
\end{equation}

乘法交换律
\begin{equation}
    fg=gf
\end{equation}

乘法结合律
\begin{equation}
    \pth{fg}h=f\pth{gh}
\end{equation}

乘法对加法的分配律
\begin{equation}
    f\pth{g+h}=fg+fh
\end{equation}

等比数列求和公式:对于多项式$f$和$g\neq 1$
\begin{equation}
    f+gf+fg^2+\cdots +fg^{n-1}=\sum_{i=0}^{n-1}fg^{i}=\frac{f-fg^{n}}{1-g}
\end{equation}
\subsubsection{总结}
在$\GF\pth{2}$这个有限域上的运算和我们在实数集上的运算很不一样,所以在后文中我们应该着重注意运算是定义在$\GF\pth{2}$上的还是定义在实数集上的。同时,我们也该清楚定理叙述的是多项式之间的关系还是值之间的关系。
\section{流密码的基本概念}
之前我们讲到“一次一密”的加密方式是完善保密,同时,“一次一密”的缺点也十分显著:密钥过长。那么,有什么办法能规避这样的缺点呢?事实上,“一次一密”的加密方式之所以是完善保密的,最重要的一点是每次密钥的字符串是随机生成的。通过之前讲的香农定理我们可以知道,每个密钥字符串生成的概率均是相等的。如果我们可以降低一点这种随机性的要求,那也许就能实现短密钥+强保密的目标。\par
为此,我们引入\textbf{伪随机数序列}的概念。由于这个概念的严格定义需要高超的概率论及算法知识,我们只需要感性地理解伪随机数序列为一种,由确定的算法产生的(即相同的初始条件下的输出是相同的),与真随机数序列性质几乎一样的序列。\par
那么\textbf{流密码}的工作模式可以简单地看作:对于给定长度的一串明文$M=m_1m_2\cdots m_n$, 我们输入密钥$k$, 通过某种算法产生一个同样长度的伪随机序列$Z=z_1z_2\cdots z_n$作为密钥流, 输出结果$Y=y_1y_2\cdots y_n$为明文串与伪随机序列按位异或的结果$y_i=m_i\oplus z_i$. 根据之前在$\GF\pth{2}$上的讨论,解密算法也是将密文串与密钥流进行按位异或,即$m_i=y_i\oplus z_i$.\par
根据上述的定义,流密码与一次一密的区别就在于,在与明文串进行按位异或的过程中,一次一密使用的是真随机序列,流密码使用的是伪随机序列。如何能使伪随机序列的表现足够像真随机序列,则是流密码安全性的关键。因此,对流密码的研究,主要就在于产生伪随机序列的算法上。\par
设$f(k, \sigma_i)$为一个能产生伪随机序列的算法,其中$k$为输入的密钥,$\sigma_i$为当前时刻系统的状态。在每个时刻,$f(k, \sigma_i)$输出一个伪随机数,同时系统状态改变为$\sigma_{i+1}$. 常把一个用于加密算法的伪随机序列称为\textbf{密钥流},产生伪随机序列的算法称为\textbf{密钥流产生器}。\par
那么流密码的过程可以理解为:如果设明文为二进制串$X=x_1x_2\cdots x_n$, 密钥为$k$. 在初始状态下,输入$x_1$和$k$, 密钥流生成器根据当前的状态输出一个伪随机数$z_1$, 输出密文$y_1=x_1\oplus z_1$. 接着输入$x_2$, 密钥流生成器根据当前的状态输出一个伪随机数$z_2$, 输出密文$y_2=x_2\oplus z_2$. 以此类推。
\section{密钥流产生器}
之前在讲到密钥流生成器的时候,我们提到了系统当前的状态$\sigma_i$. 这里的状态,是根据当前的输入和输出而变化的。每输入一个数,密钥流生成器每输出一个数,当前系统的状态就发生了变化。之后会介绍一些具体的例子让大家更加理解“系统的状态”的含义。在这里,我们将流密码分为同步流密码和自同步流密码。如果密钥流生成器中的状态与输入的明文有关,则称为\textbf{自同步流密码},反之则称为\textbf{同步流密码}。我们接下来讨论的都是同步流密码。\par
为了生成近似真随机的伪随机序列,常见的密钥流产生器由线性部分和非线性部分组合。但是,并非所有的密钥流产生器都是这种工作原理。还有别的能产生伪随机序列密钥流的方法,如RC4算法等。
\subsection{密钥流产生器的线性部分}
\subsubsection{LFSR}
根据上述的讨论,密钥流产生器应有两个功能:输出密钥流与更新系统状态。密钥流产生器的线性部分也有两个功能:输出中间的过渡字符,并且更新系统状态。所谓中间的过渡字符,意思是该部分的输出并非作为最终输出的伪随机数,而是一个中间用于之后计算的数。\par
常见的处理方法为:系统的状态$\sigma_i=a_{n, i}a_{n-1, i}\cdots a_{1, i}$由$n$个二进制数构成。在启动之前,具有初始状态$a_{n, 0}a_{n-1, 0}\cdots a_{1, 0}$. 之后,每次的输出$b_i$为
\begin{equation}
    b_i=a_{1, i-1}
\end{equation}

而状态更新的方法为
\begin{equation}
    a_{j, i}=\begin{dcases}a_{j+1, i}&1\leq j\leq n-1\\f\pth{a_{1, i-1}, a_{2, i-1}, \ldots, a_{n, i-1}}&j=n\end{dcases}
\end{equation}

其中\textbf{反馈函数}$f\pth{a_{1, i-1}, a_{2, i-1}, \ldots, a_{n, i-1}}$是一个$\GF\pth{2}$上的线性函数。即:
\begin{align}
    a_{n, i}&=f\pth{a_{1, i-1}, a_{2, i-1}, \ldots, a_{n, i-1}}\notag\\
    &=c_na_{1, i-1}+c_{n-1}a_{2, i-1}+\cdots +c_1a_{n, i-1}\\
    &=\sum_{k=1}^nc_{n+1-k}a_{k, i-1}
\end{align}
其中$c_k\in\GF\pth{2}$, 即$c_k$的取值为$0$或$1$. 这些数字都是固定的,由密钥流产生器本身决定。而这里的加法、乘法运算均为$\GF\pth{2}$上的运算。\par
我们通过一个例子来熟悉:\par
设初始状态为$110$, 反馈函数$f(a_{3, i}, a_{2, i}, a_{1, i})=a_{3, i}+a_{1, i}$. 则我们可以通过下表来了解这个线性部分的输出:
\begin{table}[H]
    \centering
    \begin{tabular}{c|c|c|c|c|c|c|c|c|c}\hline
        $i$&0&1&2&3&4&5&6&7&$\cdots$\\\hline
        $f(a_3, a_2, a_1)$&1&0&1&0&0&1&1&1&$\cdots$\\\hline
        $a_3$&1&1&0&1&0&0&1&1&$\cdots$\\\hline
        $a_2$&1&1&1&0&1&0&0&1&$\cdots$\\\hline
        $a_1$&0&1&1&1&0&1&0&0&$\cdots$\\\hline
        $b$&&0&1&1&1&0&1&0&$\cdots$\\\hline
    \end{tabular}
\end{table}

其输出序列就为$011101\quad 011101\cdots$.\par
我们由上面的表可以发现,从每一列来看,随着$i$的递增,上一行的数会传给下一行的数。这似乎是某种线性寄存器的工作形式。因此,我们称密钥流产生器的线性部分为一个\textbf{线性反馈移位寄存器}(Linear Feedback Shift Register, LFSR). 若其状态具有$n$个二进制数字,且$c_n\neq 0$, 则称其为$n$级LFSR.
\subsubsection{LFSR的输出序列}
如果设$B=b_1b_2\cdots b_i\cdots $表示LFSR的输出序列,则由上述的讨论也可以知道
\begin{equation}
    \pth{b_{i+1}b_{i+2}\cdots b_{i+n}}=\pth{a_{1, i-1}a_{2, i-1}\cdots a_{n, i-1}}
\end{equation}

我们还可以认为,每一个$b_i$都可以代表LFSR的一个状态。那么,由于LFSR的一个状态中有$n$个二进制数,故一个LFSR至多有$2^n$种状态。如果称状态$\sigma=00\cdots 0$为零状态,下面我们证明:零状态的下一个状态是零状态,任何非零状态的下一个状态均不是零状态。
\begin{prove}
前半句话显然,设$\forall 1\leq j\leq n, a_{j, i}=0$. 对于$j\leq n-1$, $a_{j, i+1}=a_{j+1, i}=0$. 而
\[a_{n, i}=c_na_{1, i-1}+c_{n-1}a_{2, i-1}+\cdots +c_1a_{n, i-1}=0\]

对于后半句话,若某个非零状态$\sigma_i=a_{n, i}a_{n-1, i}\cdots a_{1, i}$的下一个状态为零状态,则由定义可知,$\forall 2\leq j\leq n, a_{j, i}=a_{j-1, i+1}=0, a_{n, i+1}\neq 0$. 且
\[0=a_{n, i+1}=c_na_{1, i}+c_{n-1}a_{2, i}+\cdots +c_1a_{n, i}=c_na_{1, i}\]

又由于$c_n\neq 0$, 故$a_{1, i}=0$矛盾。
\end{prove}

由上述讨论可知,一个LFSR中的状态,至多$2^n-1$个之后即达成循环。也就是说,一个LFSR产生的序列,周期至多为$2^n-1$. 
\begin{Definition}
    我们称由LFSR产生的周期为$2^n-1$的序列为一个$m$序列。
\end{Definition}
\subsubsection{$m$序列相关定理}
上述的讨论中,我们提到,反馈函数
\[a_{n, i}=c_na_{1, i-1}+c_{n-1}a_{2, i-1}+\cdots +c_1a_{n, i-1}\]
中的$c_1, c_2,\ldots ,c_n$由LFSR本身决定。因此:
\begin{Definition}
    我们称$\GF\pth{2}$上的多项式
\begin{equation}
    p(x)=1+c_1x+\cdots +c_{n-1}x^{n-1}+c_nx^n
\end{equation}
为LFSR的特征多项式。\par
对于LFSR生成的一个序列$a_1a_2\cdots a_n\cdots$, 称幂级数
\begin{equation}
    A\pth{x}=\sum_{i=1}^{\infty}a_ix^{i-1}
\end{equation}
为该序列的生成函数。\par
对于使用给定的LFSR, 由于初始状态不同而产生的所有$2^n-1$个非零序列构成的集合记作$G\pth{p\pth{x}}$.
\end{Definition}

下面叙述一个在证明中很有用的定理:
\begin{theorem}
设LFSR的特征多项式$p(x)=1+c_1x+\cdots +c_{n-1}x^{n-1}+c_nx^n$, $A\pth{x}$为$G\pth{p(x)}$中任意序列$\brace{a_n}$的生成函数,则对于$\GF\pth{2}$上的多项式$p(x)$和$A(x)$, 满足
\begin{equation}
    A(x)=\frac{\phi(x)}{p(x)}
\end{equation}

其中
\begin{equation}
    \phi(x)=\sum_{i=1}^n\pth{c_{n-i}x^{n-i}\sum_{j=1}^ia_jx^{j-1}}
\end{equation}
\end{theorem}

我们可以对$\phi(x)$进行展开:
\begin{align*}
    \phi(x)&=\sum_{i=1}^n\pth{c_{n-i}x^{n-i}\sum_{j=1}^ia_jx^{j-1}}\\
    &=\sum_{i=1}^n\pth{\sum_{j=1}^ic_{n-i}a_jx^{n-i+j-1}}\\
    &\xlongequal{k=n-i+j-1}\sum_{i=1}^n\sum_{k=n-i}^{n-1}c_{n-i}a_{k-n+i+1}x^k
\end{align*}

因此,我们可以发现,$\phi(x)$的次数不超过$n-1$.
\begin{Definition}
对于$\GF\pth{2}$上的多项式$p(x)$, 若$p(x)\mid x^p-1$, 则称最小的$p$为$p(x)$的阶。
\end{Definition}

一个与之相关的定理是:
\begin{theorem}
设$p(x)$是$n$次不可约多项式,其阶为$m$. 则$\forall \brace{a_n}\in G\pth{p(x)}$, $\brace{a_n}$的周期为$m$.
\end{theorem}
\begin{Definition}
若$n$次不可约多项式$p(x)$的阶为$2^{n}-1$, 则称$p(x)$是本原多项式。
\end{Definition}

下面叙述的是最关键的一个定理:
\begin{theorem}
设$\brace{a_n}\in G\pth{p(x)}$, 则$\brace{a_n}$为$m$序列的充要条件是$p(x)$是本原多项式。
\end{theorem}
\subsubsection{LFSR是伪随机数产生器}
为了叙述本节的定理,我们引入两个概念:
\begin{Definition}
对于序列$\brace{a_n}$, 长度最大为$n$的连续的0或者1称为一个长度为$n$的0游程或1游程。\par
对于$\GF\pth{2}$上周期为$2$的序列$\brace{a_n}$, 称其异相关函数为
\begin{equation}
    R\pth{\tau}=\frac{1}{T}\sum_{k=1}^T\pth{-1}^{a_k}\pth{-1}^{a_{k+\tau}}, 0<\tau\leq T-1
\end{equation}
\end{Definition}

之前我们提到的伪随机数序列,我们在这里给出一种定义方法:
\begin{Definition}
对于周期序列$\brace{a_n}$, 其为伪随机序列的条件为
\begin{enumerate}
    \item 在序列的一个周期内,$0$与$1$的个数至多相差1.
    \item 在序列的一个周期内,长为$i$的游程占游程总数的$\ext\frac{1}{2^i}$, 且其中0游程与1游程个数相等。
    \item 该序列的异相关函数是个常数。
\end{enumerate}
\end{Definition}

那么我们可以证明,一个$n$长$m$序列是这种意义下的伪随机序列。
\subsubsection{LFSR密码的破译}
在本节介绍的破译方式中,值得强调的是,敌手是知道LFSR为$n$级,且敌手获得了一串长度为$2n$的明密文对。\par
假设敌手获得的明密文对为$x_1x_2\cdots x_{2n}$和$y_1y_2\cdots y_{2n}$, 其需要破译的是LFSR的特征多项式的系数$c_1, c_2,\ldots, c_n$. 那么由于在$\GF\pth{2}$上
\[y_i=x_i+z_i\]
其中$z_i$为产生的密钥流的第$i$位。\par
故在$\GF\pth{2}$上
\[x_i+y_i=x_i+x_i+z_i=z_i\]

从而敌手就获得了一段长度为$2n$的密钥流$z_1z_2\cdots z_{2n}$.\par
如果记
\begin{equation}
    S_i=\pth{z_{i+1}, z_{i+2}, \ldots, z_{i+n}}^{\mathrm{T}}, i=0, 1,\ldots, n-1
\end{equation}

且
\begin{equation}
    X=\pth{S_0,S_1,\ldots, S_{n-1}}
\end{equation}

故根据表达式
\begin{equation}
    z_{n+i}=c_nz_{i}+c_{n-1}z_{i+1}+\cdots +c_{1}z_{i+n-1}
\end{equation}

可得:
\begin{equation}
    \pth{z_{n+1}, z_{n+2},\ldots, z_{2n}}=\pth{c_n, c_{n-1}, \cdots ,c_{1}}X
\end{equation}

而我们可以证明$X$是可逆的。故
\begin{equation}
    \pth{c_n, c_{n-1}, \cdots ,c_{1}}=\pth{z_{n+1}, z_{n+2},\ldots, z_{2n}}X^{-1}
\end{equation}
\subsection{密钥流产生器的非线性部分}
由上述的讨论我们可以发现,线性的密钥流产生器产生的密钥流一定是周期的。而周期较短的密钥流则相当于一种多表代换密码,相对容易破解。因此,只有加上非线性的部分来处理,才能提高安全性。\par
密钥流产生器的非线性部分的主要工作是接受一个或多个LFSR的输入,以非线性的方式,输出密钥流。\par
我们综合判定一个输出密钥流的优劣,可以从两个方面:周期和线性复杂度。周期越长,线性复杂度越高的密钥流越安全。周期我们可以直接衡量,而线性复杂度我们该如何衡量呢?我们称一个序列的线性复杂度为生成该序列的最短LFSR的级数。即若该序列周期$T$满足$2^{n-1}-1<T\leq 2^{n}-1$, 则该序列的线性复杂度为$n$.
\subsubsection{Geffe序列生成器}
Geffe序列生成器接受三个LFSR的输入,其输入分别为序列$\brace{a_n^{(1)}}, \brace{a_n^{(2)}}, \brace{a_n^{(3)}}$.\par
其输出序列$\brace{b_k}$可以表示为
\begin{table}[H]
    \centering
    \begin{tabular}{c|c}\hline
        $a_k^{(2)}$&$b_k$\\\hline
        $0$&$a_k^{(3)}$\\\hline
        $1$&$a_k^{(1)}$\\\hline
    \end{tabular}
\end{table}

若$\brace{a_n^{(1)}}$, $\brace{a_n^{(2)}}$和$\brace{a_n^{(3)}}$的周期分别为$2^{n_1}-1$, $2^{n_2}-1$和$2^{n_3}-1$且$n_1, n_2, n_3$两两互素,则$\brace{b_k}$的周期为$\pth{2^{n_1}-1}\pth{2^{n_2}-1}\pth{2^{n_3}-1}$, 线性复杂度为$\pth{n_1+n_3}n_2+n_3$
\subsubsection{JK触发器}
JK触发器接受两个LFSR的输入,其输入分别为序列$\brace{a_n^{(1)}}, \brace{a_n^{(2)}}$.\par
其输出序列$\brace{b_k}$可以表示为
\begin{table}[H]
    \centering
    \begin{tabular}{c|c|c}\hline
        $a_k^{(1)}$&$a_k^{(2)}$&$b_k$\\\hline
        $0$&$0$&$b_{k-1}$\\\hline
        $0$&$1$&0\\\hline
        $1$&$0$&0\\\hline
        $1$&$1$&$\overline{b_{k-1}}$\\\hline
    \end{tabular}
\end{table}

若$\brace{a_n^{(1)}}$和$\brace{a_n^{(2)}}$的周期分别为$2^n-1$和$2^m-1$且$m, n$互素,$a_0^{(1)}+a_0^{(2)}=1$, 则$\brace{b_k}$的周期为$\pth{2^n-1}\pth{2^m-1}$
\subsubsection{钟控序列生成器}
钟控序列接受两个LFSR的输入,其输入分别为序列$\brace{a_n^{(1)}}, \brace{a_n^{(2)}}$. 前一个序列控制后一个序列的时钟周期$n_k$.\par
其输出序列$\brace{b_k}$可以表示为
\begin{table}[H]
    \centering
    \begin{tabular}{c|c|c}\hline
        $a_k^{(1)}$&$n_k$&$b_k$\\\hline
        $0$&$n_{k-1}$&$a_{n_k}^{(2)}$\\\hline
        $1$&$n_{k-1}+1$&$a_{n_k}^{(2)}$\\\hline
    \end{tabular}
\end{table}

若$\brace{a_n^{(1)}}$和$\brace{a_n^{(2)}}$的周期分别为$p_1$和$p_2$, 且记$\ext w=\sum_{i=0}^{p_1-1}a_i^{(1)}$, 则$\brace{b_k}$的周期为$\ext\frac{p_1p_2}{\gcd\pth{w, p_2}}$. 若$p_1=2^m-1, p_2=2^n-1$, 则其线性复杂度为$n\pth{2^m-1}$.